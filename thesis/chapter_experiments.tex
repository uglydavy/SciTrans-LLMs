% ============================================================================
% SciTrans-LLMs Thesis Chapter: Experimental Evaluation
% ============================================================================
% 
% This is a template for your thesis experiments chapter.
% It includes placeholders for all generated tables and figures.
%
% To use:
%   1. Run experiments: python scripts/full_pipeline.py --backend openai
%   2. Copy results/thesis/*.tex to your thesis folder
%   3. Include this chapter: % ============================================================================
% SciTrans-LLMs Thesis Chapter: Experimental Evaluation
% ============================================================================
% 
% This is a template for your thesis experiments chapter.
% It includes placeholders for all generated tables and figures.
%
% To use:
%   1. Run experiments: python scripts/full_pipeline.py --backend openai
%   2. Copy results/thesis/*.tex to your thesis folder
%   3. Include this chapter: % ============================================================================
% SciTrans-LLMs Thesis Chapter: Experimental Evaluation
% ============================================================================
% 
% This is a template for your thesis experiments chapter.
% It includes placeholders for all generated tables and figures.
%
% To use:
%   1. Run experiments: python scripts/full_pipeline.py --backend openai
%   2. Copy results/thesis/*.tex to your thesis folder
%   3. Include this chapter: % ============================================================================
% SciTrans-LLMs Thesis Chapter: Experimental Evaluation
% ============================================================================
% 
% This is a template for your thesis experiments chapter.
% It includes placeholders for all generated tables and figures.
%
% To use:
%   1. Run experiments: python scripts/full_pipeline.py --backend openai
%   2. Copy results/thesis/*.tex to your thesis folder
%   3. Include this chapter: \include{chapter_experiments}
%
% ============================================================================

\chapter{Experimental Evaluation}
\label{ch:experiments}

This chapter presents the experimental evaluation of our document translation system. We describe our experimental setup, present main results, analyze component contributions through ablation studies, and compare with baseline systems.

% ============================================================================
\section{Experimental Setup}
\label{sec:setup}
% ============================================================================

\subsection{Dataset}

We evaluate our system on a corpus of scientific abstracts in the English-French language pair. The corpus consists of:

\begin{itemize}
    \item \textbf{Source}: English scientific abstracts from [describe your sources]
    \item \textbf{Reference}: Professional French translations
    \item \textbf{Size}: [N] documents, [M] segments
    \item \textbf{Domain}: Machine learning, natural language processing
\end{itemize}

% If you have corpus statistics, uncomment and fill in:
% \begin{table}[h]
% \centering
% \begin{tabular}{lc}
% \toprule
% \textbf{Statistic} & \textbf{Value} \\
% \midrule
% Total documents & XXX \\
% Total segments & XXX \\
% Average segments per document & X.X \\
% Average source length (tokens) & XXX \\
% \bottomrule
% \end{tabular}
% \caption{Corpus statistics.}
% \label{tab:corpus-stats}
% \end{table}

\subsection{Evaluation Metrics}

We use the following metrics to evaluate translation quality:

\begin{itemize}
    \item \textbf{BLEU} \cite{papineni2002bleu}: Measures n-gram overlap with reference translations
    \item \textbf{chrF++} \cite{popovic2017chrf}: Character-level F-score, more robust for morphologically rich languages
    \item \textbf{Glossary Adherence}: Percentage of domain terms correctly translated according to our glossary
    \item \textbf{Placeholder Preservation}: Percentage of protected content (formulas, URLs) preserved intact
\end{itemize}

\subsection{System Configuration}

Our system was configured with the following settings:

\begin{itemize}
    \item \textbf{Translation backend}: [GPT-4 / DeepSeek / etc.]
    \item \textbf{Glossary}: [N] domain-specific term pairs
    \item \textbf{Context window}: [N] previous segments
    \item \textbf{Refinement}: Glossary enforcement + coherence pass
\end{itemize}

% ============================================================================
\section{Main Results}
\label{sec:results}
% ============================================================================

Table~\ref{tab:main-results} presents the main translation quality results comparing our full system with baseline configurations.

% Include generated table
\input{results/thesis/results_table}

Our system achieves [X.X] BLEU and [X.X] chrF++ scores, representing an improvement of [Y.Y] BLEU points over the baseline. Notably, glossary adherence reaches [Z]\%, demonstrating effective terminology control.

\subsection{Qualitative Analysis}

Beyond automatic metrics, we observe several qualitative improvements:

\begin{enumerate}
    \item \textbf{Consistent terminology}: Technical terms are translated consistently throughout documents
    \item \textbf{Formula preservation}: Mathematical expressions are preserved without corruption
    \item \textbf{Coherent references}: Pronouns and demonstratives maintain correct referents across sentences
\end{enumerate}

% Example translations
% \begin{table}[h]
% \centering
% \small
% \begin{tabular}{p{0.45\textwidth}p{0.45\textwidth}}
% \toprule
% \textbf{Source} & \textbf{Translation} \\
% \midrule
% Deep learning has revolutionized natural language processing. & L'apprentissage profond a révolutionné le traitement automatique du langage naturel. \\
% \bottomrule
% \end{tabular}
% \caption{Example translation showing terminology consistency.}
% \label{tab:example-translation}
% \end{table}

% ============================================================================
\section{Ablation Study}
\label{sec:ablation}
% ============================================================================

To understand the contribution of each component, we conduct an ablation study by systematically disabling individual features.

% Include generated ablation table
\input{results/thesis/ablation_table}

\subsection{Component Analysis}

Table~\ref{tab:contribution} quantifies each component's contribution to overall quality improvement.

% Include generated contribution table
\input{results/thesis/contribution_table}

Key findings:

\begin{itemize}
    \item \textbf{Glossary enforcement} contributes the largest improvement (+X.X BLEU), confirming the importance of terminology control for scientific translation
    \item \textbf{Document-level refinement} provides +X.X BLEU improvement by improving coherence
    \item \textbf{Placeholder masking} ensures formula preservation with minimal impact on BLEU
    \item \textbf{Context window} shows modest but consistent improvements in pronoun resolution
\end{itemize}

% ============================================================================
\section{Baseline Comparison}
\label{sec:baselines}
% ============================================================================

We compare our system with existing translation approaches:

\begin{itemize}
    \item \textbf{Google Translate}: Commercial neural MT system
    \item \textbf{Naive LLM}: GPT-4 without glossary or context
    \item \textbf{PDFMathTranslate}: Open-source layout-preserving translator
\end{itemize}

% Include baseline comparison if available
% \input{results/thesis/baselines_table}

Our system outperforms all baselines on glossary adherence while maintaining competitive BLEU scores. The improvement is particularly notable for:

\begin{enumerate}
    \item Technical terminology: [X]\% vs [Y]\% for baseline
    \item Formula preservation: [X]\% vs [Y]\% for baseline
    \item Cross-sentence coherence: [qualitative observation]
\end{enumerate}

% ============================================================================
\section{Error Analysis}
\label{sec:errors}
% ============================================================================

We analyze common error types in our system's output:

\subsection{Terminology Errors}

Despite glossary enforcement, some terms are missed when:
\begin{itemize}
    \item Terms appear in unusual grammatical forms
    \item Context suggests alternative translations
    \item Terms are part of larger phrases not in the glossary
\end{itemize}

\subsection{Coherence Issues}

Remaining coherence issues include:
\begin{itemize}
    \item Long-range dependencies beyond context window
    \item Implicit subjects in complex sentences
    \item Domain-specific discourse conventions
\end{itemize}

% ============================================================================
\section{Discussion}
\label{sec:discussion}
% ============================================================================

Our experiments demonstrate that:

\begin{enumerate}
    \item \textbf{Terminology control is crucial}: Glossary enforcement provides the largest quality improvement for scientific translation
    
    \item \textbf{Document context matters}: Even simple context windows improve coherence, with larger windows providing diminishing returns
    
    \item \textbf{LLM refinement is effective}: Post-translation refinement catches errors missed during initial translation
    
    \item \textbf{Layout preservation is achievable}: Placeholder masking successfully protects formulas and special content
\end{enumerate}

\subsection{Limitations}

Our approach has several limitations:

\begin{itemize}
    \item Requires curated glossaries for optimal performance
    \item LLM API costs may limit large-scale deployment
    \item PDF reconstruction quality depends on source document complexity
\end{itemize}

\subsection{Future Work}

Promising directions for future research include:

\begin{itemize}
    \item Automatic glossary extraction from parallel corpora
    \item Fine-tuning smaller models for specific domains
    \item Multi-document translation with shared context
    \item Interactive refinement with human feedback
\end{itemize}

% ============================================================================
\section{Summary}
\label{sec:summary}
% ============================================================================

This chapter presented a comprehensive evaluation of our document translation system. Key findings include:

\begin{itemize}
    \item Our system achieves [X.X] BLEU and [Z]\% glossary adherence on scientific abstracts
    \item Ablation studies confirm the contribution of each component
    \item Comparison with baselines demonstrates improvements in terminology control
\end{itemize}

These results support our thesis contributions:
\begin{enumerate}
    \item Terminology-constrained translation via glossary prompting and enforcement
    \item Document-level coherence through context windows and refinement
    \item Research-grade evaluation framework for systematic analysis
\end{enumerate}


%
% ============================================================================

\chapter{Experimental Evaluation}
\label{ch:experiments}

This chapter presents the experimental evaluation of our document translation system. We describe our experimental setup, present main results, analyze component contributions through ablation studies, and compare with baseline systems.

% ============================================================================
\section{Experimental Setup}
\label{sec:setup}
% ============================================================================

\subsection{Dataset}

We evaluate our system on a corpus of scientific abstracts in the English-French language pair. The corpus consists of:

\begin{itemize}
    \item \textbf{Source}: English scientific abstracts from [describe your sources]
    \item \textbf{Reference}: Professional French translations
    \item \textbf{Size}: [N] documents, [M] segments
    \item \textbf{Domain}: Machine learning, natural language processing
\end{itemize}

% If you have corpus statistics, uncomment and fill in:
% \begin{table}[h]
% \centering
% \begin{tabular}{lc}
% \toprule
% \textbf{Statistic} & \textbf{Value} \\
% \midrule
% Total documents & XXX \\
% Total segments & XXX \\
% Average segments per document & X.X \\
% Average source length (tokens) & XXX \\
% \bottomrule
% \end{tabular}
% \caption{Corpus statistics.}
% \label{tab:corpus-stats}
% \end{table}

\subsection{Evaluation Metrics}

We use the following metrics to evaluate translation quality:

\begin{itemize}
    \item \textbf{BLEU} \cite{papineni2002bleu}: Measures n-gram overlap with reference translations
    \item \textbf{chrF++} \cite{popovic2017chrf}: Character-level F-score, more robust for morphologically rich languages
    \item \textbf{Glossary Adherence}: Percentage of domain terms correctly translated according to our glossary
    \item \textbf{Placeholder Preservation}: Percentage of protected content (formulas, URLs) preserved intact
\end{itemize}

\subsection{System Configuration}

Our system was configured with the following settings:

\begin{itemize}
    \item \textbf{Translation backend}: [GPT-4 / DeepSeek / etc.]
    \item \textbf{Glossary}: [N] domain-specific term pairs
    \item \textbf{Context window}: [N] previous segments
    \item \textbf{Refinement}: Glossary enforcement + coherence pass
\end{itemize}

% ============================================================================
\section{Main Results}
\label{sec:results}
% ============================================================================

Table~\ref{tab:main-results} presents the main translation quality results comparing our full system with baseline configurations.

% Include generated table
% Main results table
\begin{table}[t]
\centering
\small
\begin{tabular}{@{}lcccc@{}}
\toprule
\textbf{System} & \textbf{BLEU}$\uparrow$ & \textbf{chrF++}$\uparrow$ & \textbf{Gloss.}$\uparrow$ & \textbf{Time} \\
\midrule
\textbf{Scitrans Llms} & \textbf{12.0} & \textbf{41.5} & \textbf{100\%} & 1.3s \\
\textbf{Full} & \textbf{12.0} & \textbf{41.5} & \textbf{100\%} & 1.1s \\
No Glossary & 13.2 & 38.4 & 10\% & 0.0s \\
No Context & 12.0 & 41.5 & 100\% & 1.1s \\
No Refinement & 13.2 & 38.4 & 10\% & 0.0s \\
No Masking & 12.0 & 41.5 & 100\% & 1.1s \\
Minimal & 13.2 & 38.4 & 10\% & 0.0s \\
\bottomrule
\end{tabular}
\caption{Translation quality results on the test corpus. Best results in bold.}
\label{tab:main-results}
\end{table}

Our system achieves [X.X] BLEU and [X.X] chrF++ scores, representing an improvement of [Y.Y] BLEU points over the baseline. Notably, glossary adherence reaches [Z]\%, demonstrating effective terminology control.

\subsection{Qualitative Analysis}

Beyond automatic metrics, we observe several qualitative improvements:

\begin{enumerate}
    \item \textbf{Consistent terminology}: Technical terms are translated consistently throughout documents
    \item \textbf{Formula preservation}: Mathematical expressions are preserved without corruption
    \item \textbf{Coherent references}: Pronouns and demonstratives maintain correct referents across sentences
\end{enumerate}

% Example translations
% \begin{table}[h]
% \centering
% \small
% \begin{tabular}{p{0.45\textwidth}p{0.45\textwidth}}
% \toprule
% \textbf{Source} & \textbf{Translation} \\
% \midrule
% Deep learning has revolutionized natural language processing. & L'apprentissage profond a révolutionné le traitement automatique du langage naturel. \\
% \bottomrule
% \end{tabular}
% \caption{Example translation showing terminology consistency.}
% \label{tab:example-translation}
% \end{table}

% ============================================================================
\section{Ablation Study}
\label{sec:ablation}
% ============================================================================

To understand the contribution of each component, we conduct an ablation study by systematically disabling individual features.

% Include generated ablation table
% Ablation study table
\begin{table}[t]
\centering
\small
\begin{tabular}{@{}lccc@{}}
\toprule
\textbf{Configuration} & \textbf{BLEU} & \textbf{$\Delta$BLEU} & \textbf{Gloss.} \\
\midrule
\textbf{Full System} & \textbf{12.0} & --- & \textbf{100\%} \\
-- Glossary & 13.2 & +1.3 & 10\% \\
-- Context & 12.0 & +0.0 & 100\% \\
-- Refinement & 13.2 & +1.3 & 10\% \\
-- Masking & 12.0 & +0.0 & 100\% \\
Minimal & 13.2 & +1.3 & 10\% \\
\bottomrule
\end{tabular}
\caption{Ablation study results. $\Delta$BLEU shows difference from full system.}
\label{tab:ablation}
\end{table}

\subsection{Component Analysis}

Table~\ref{tab:contribution} quantifies each component's contribution to overall quality improvement.

% Include generated contribution table
\input{results/thesis/contribution_table}

Key findings:

\begin{itemize}
    \item \textbf{Glossary enforcement} contributes the largest improvement (+X.X BLEU), confirming the importance of terminology control for scientific translation
    \item \textbf{Document-level refinement} provides +X.X BLEU improvement by improving coherence
    \item \textbf{Placeholder masking} ensures formula preservation with minimal impact on BLEU
    \item \textbf{Context window} shows modest but consistent improvements in pronoun resolution
\end{itemize}

% ============================================================================
\section{Baseline Comparison}
\label{sec:baselines}
% ============================================================================

We compare our system with existing translation approaches:

\begin{itemize}
    \item \textbf{Google Translate}: Commercial neural MT system
    \item \textbf{Naive LLM}: GPT-4 without glossary or context
    \item \textbf{PDFMathTranslate}: Open-source layout-preserving translator
\end{itemize}

% Include baseline comparison if available
% \input{results/thesis/baselines_table}

Our system outperforms all baselines on glossary adherence while maintaining competitive BLEU scores. The improvement is particularly notable for:

\begin{enumerate}
    \item Technical terminology: [X]\% vs [Y]\% for baseline
    \item Formula preservation: [X]\% vs [Y]\% for baseline
    \item Cross-sentence coherence: [qualitative observation]
\end{enumerate}

% ============================================================================
\section{Error Analysis}
\label{sec:errors}
% ============================================================================

We analyze common error types in our system's output:

\subsection{Terminology Errors}

Despite glossary enforcement, some terms are missed when:
\begin{itemize}
    \item Terms appear in unusual grammatical forms
    \item Context suggests alternative translations
    \item Terms are part of larger phrases not in the glossary
\end{itemize}

\subsection{Coherence Issues}

Remaining coherence issues include:
\begin{itemize}
    \item Long-range dependencies beyond context window
    \item Implicit subjects in complex sentences
    \item Domain-specific discourse conventions
\end{itemize}

% ============================================================================
\section{Discussion}
\label{sec:discussion}
% ============================================================================

Our experiments demonstrate that:

\begin{enumerate}
    \item \textbf{Terminology control is crucial}: Glossary enforcement provides the largest quality improvement for scientific translation
    
    \item \textbf{Document context matters}: Even simple context windows improve coherence, with larger windows providing diminishing returns
    
    \item \textbf{LLM refinement is effective}: Post-translation refinement catches errors missed during initial translation
    
    \item \textbf{Layout preservation is achievable}: Placeholder masking successfully protects formulas and special content
\end{enumerate}

\subsection{Limitations}

Our approach has several limitations:

\begin{itemize}
    \item Requires curated glossaries for optimal performance
    \item LLM API costs may limit large-scale deployment
    \item PDF reconstruction quality depends on source document complexity
\end{itemize}

\subsection{Future Work}

Promising directions for future research include:

\begin{itemize}
    \item Automatic glossary extraction from parallel corpora
    \item Fine-tuning smaller models for specific domains
    \item Multi-document translation with shared context
    \item Interactive refinement with human feedback
\end{itemize}

% ============================================================================
\section{Summary}
\label{sec:summary}
% ============================================================================

This chapter presented a comprehensive evaluation of our document translation system. Key findings include:

\begin{itemize}
    \item Our system achieves [X.X] BLEU and [Z]\% glossary adherence on scientific abstracts
    \item Ablation studies confirm the contribution of each component
    \item Comparison with baselines demonstrates improvements in terminology control
\end{itemize}

These results support our thesis contributions:
\begin{enumerate}
    \item Terminology-constrained translation via glossary prompting and enforcement
    \item Document-level coherence through context windows and refinement
    \item Research-grade evaluation framework for systematic analysis
\end{enumerate}


%
% ============================================================================

\chapter{Experimental Evaluation}
\label{ch:experiments}

This chapter presents the experimental evaluation of our document translation system. We describe our experimental setup, present main results, analyze component contributions through ablation studies, and compare with baseline systems.

% ============================================================================
\section{Experimental Setup}
\label{sec:setup}
% ============================================================================

\subsection{Dataset}

We evaluate our system on a corpus of scientific abstracts in the English-French language pair. The corpus consists of:

\begin{itemize}
    \item \textbf{Source}: English scientific abstracts from [describe your sources]
    \item \textbf{Reference}: Professional French translations
    \item \textbf{Size}: [N] documents, [M] segments
    \item \textbf{Domain}: Machine learning, natural language processing
\end{itemize}

% If you have corpus statistics, uncomment and fill in:
% \begin{table}[h]
% \centering
% \begin{tabular}{lc}
% \toprule
% \textbf{Statistic} & \textbf{Value} \\
% \midrule
% Total documents & XXX \\
% Total segments & XXX \\
% Average segments per document & X.X \\
% Average source length (tokens) & XXX \\
% \bottomrule
% \end{tabular}
% \caption{Corpus statistics.}
% \label{tab:corpus-stats}
% \end{table}

\subsection{Evaluation Metrics}

We use the following metrics to evaluate translation quality:

\begin{itemize}
    \item \textbf{BLEU} \cite{papineni2002bleu}: Measures n-gram overlap with reference translations
    \item \textbf{chrF++} \cite{popovic2017chrf}: Character-level F-score, more robust for morphologically rich languages
    \item \textbf{Glossary Adherence}: Percentage of domain terms correctly translated according to our glossary
    \item \textbf{Placeholder Preservation}: Percentage of protected content (formulas, URLs) preserved intact
\end{itemize}

\subsection{System Configuration}

Our system was configured with the following settings:

\begin{itemize}
    \item \textbf{Translation backend}: [GPT-4 / DeepSeek / etc.]
    \item \textbf{Glossary}: [N] domain-specific term pairs
    \item \textbf{Context window}: [N] previous segments
    \item \textbf{Refinement}: Glossary enforcement + coherence pass
\end{itemize}

% ============================================================================
\section{Main Results}
\label{sec:results}
% ============================================================================

Table~\ref{tab:main-results} presents the main translation quality results comparing our full system with baseline configurations.

% Include generated table
% Main results table
\begin{table}[t]
\centering
\small
\begin{tabular}{@{}lcccc@{}}
\toprule
\textbf{System} & \textbf{BLEU}$\uparrow$ & \textbf{chrF++}$\uparrow$ & \textbf{Gloss.}$\uparrow$ & \textbf{Time} \\
\midrule
\textbf{Scitrans Llms} & \textbf{12.0} & \textbf{41.5} & \textbf{100\%} & 1.3s \\
\textbf{Full} & \textbf{12.0} & \textbf{41.5} & \textbf{100\%} & 1.1s \\
No Glossary & 13.2 & 38.4 & 10\% & 0.0s \\
No Context & 12.0 & 41.5 & 100\% & 1.1s \\
No Refinement & 13.2 & 38.4 & 10\% & 0.0s \\
No Masking & 12.0 & 41.5 & 100\% & 1.1s \\
Minimal & 13.2 & 38.4 & 10\% & 0.0s \\
\bottomrule
\end{tabular}
\caption{Translation quality results on the test corpus. Best results in bold.}
\label{tab:main-results}
\end{table}

Our system achieves [X.X] BLEU and [X.X] chrF++ scores, representing an improvement of [Y.Y] BLEU points over the baseline. Notably, glossary adherence reaches [Z]\%, demonstrating effective terminology control.

\subsection{Qualitative Analysis}

Beyond automatic metrics, we observe several qualitative improvements:

\begin{enumerate}
    \item \textbf{Consistent terminology}: Technical terms are translated consistently throughout documents
    \item \textbf{Formula preservation}: Mathematical expressions are preserved without corruption
    \item \textbf{Coherent references}: Pronouns and demonstratives maintain correct referents across sentences
\end{enumerate}

% Example translations
% \begin{table}[h]
% \centering
% \small
% \begin{tabular}{p{0.45\textwidth}p{0.45\textwidth}}
% \toprule
% \textbf{Source} & \textbf{Translation} \\
% \midrule
% Deep learning has revolutionized natural language processing. & L'apprentissage profond a révolutionné le traitement automatique du langage naturel. \\
% \bottomrule
% \end{tabular}
% \caption{Example translation showing terminology consistency.}
% \label{tab:example-translation}
% \end{table}

% ============================================================================
\section{Ablation Study}
\label{sec:ablation}
% ============================================================================

To understand the contribution of each component, we conduct an ablation study by systematically disabling individual features.

% Include generated ablation table
% Ablation study table
\begin{table}[t]
\centering
\small
\begin{tabular}{@{}lccc@{}}
\toprule
\textbf{Configuration} & \textbf{BLEU} & \textbf{$\Delta$BLEU} & \textbf{Gloss.} \\
\midrule
\textbf{Full System} & \textbf{12.0} & --- & \textbf{100\%} \\
-- Glossary & 13.2 & +1.3 & 10\% \\
-- Context & 12.0 & +0.0 & 100\% \\
-- Refinement & 13.2 & +1.3 & 10\% \\
-- Masking & 12.0 & +0.0 & 100\% \\
Minimal & 13.2 & +1.3 & 10\% \\
\bottomrule
\end{tabular}
\caption{Ablation study results. $\Delta$BLEU shows difference from full system.}
\label{tab:ablation}
\end{table}

\subsection{Component Analysis}

Table~\ref{tab:contribution} quantifies each component's contribution to overall quality improvement.

% Include generated contribution table
\input{results/thesis/contribution_table}

Key findings:

\begin{itemize}
    \item \textbf{Glossary enforcement} contributes the largest improvement (+X.X BLEU), confirming the importance of terminology control for scientific translation
    \item \textbf{Document-level refinement} provides +X.X BLEU improvement by improving coherence
    \item \textbf{Placeholder masking} ensures formula preservation with minimal impact on BLEU
    \item \textbf{Context window} shows modest but consistent improvements in pronoun resolution
\end{itemize}

% ============================================================================
\section{Baseline Comparison}
\label{sec:baselines}
% ============================================================================

We compare our system with existing translation approaches:

\begin{itemize}
    \item \textbf{Google Translate}: Commercial neural MT system
    \item \textbf{Naive LLM}: GPT-4 without glossary or context
    \item \textbf{PDFMathTranslate}: Open-source layout-preserving translator
\end{itemize}

% Include baseline comparison if available
% \input{results/thesis/baselines_table}

Our system outperforms all baselines on glossary adherence while maintaining competitive BLEU scores. The improvement is particularly notable for:

\begin{enumerate}
    \item Technical terminology: [X]\% vs [Y]\% for baseline
    \item Formula preservation: [X]\% vs [Y]\% for baseline
    \item Cross-sentence coherence: [qualitative observation]
\end{enumerate}

% ============================================================================
\section{Error Analysis}
\label{sec:errors}
% ============================================================================

We analyze common error types in our system's output:

\subsection{Terminology Errors}

Despite glossary enforcement, some terms are missed when:
\begin{itemize}
    \item Terms appear in unusual grammatical forms
    \item Context suggests alternative translations
    \item Terms are part of larger phrases not in the glossary
\end{itemize}

\subsection{Coherence Issues}

Remaining coherence issues include:
\begin{itemize}
    \item Long-range dependencies beyond context window
    \item Implicit subjects in complex sentences
    \item Domain-specific discourse conventions
\end{itemize}

% ============================================================================
\section{Discussion}
\label{sec:discussion}
% ============================================================================

Our experiments demonstrate that:

\begin{enumerate}
    \item \textbf{Terminology control is crucial}: Glossary enforcement provides the largest quality improvement for scientific translation
    
    \item \textbf{Document context matters}: Even simple context windows improve coherence, with larger windows providing diminishing returns
    
    \item \textbf{LLM refinement is effective}: Post-translation refinement catches errors missed during initial translation
    
    \item \textbf{Layout preservation is achievable}: Placeholder masking successfully protects formulas and special content
\end{enumerate}

\subsection{Limitations}

Our approach has several limitations:

\begin{itemize}
    \item Requires curated glossaries for optimal performance
    \item LLM API costs may limit large-scale deployment
    \item PDF reconstruction quality depends on source document complexity
\end{itemize}

\subsection{Future Work}

Promising directions for future research include:

\begin{itemize}
    \item Automatic glossary extraction from parallel corpora
    \item Fine-tuning smaller models for specific domains
    \item Multi-document translation with shared context
    \item Interactive refinement with human feedback
\end{itemize}

% ============================================================================
\section{Summary}
\label{sec:summary}
% ============================================================================

This chapter presented a comprehensive evaluation of our document translation system. Key findings include:

\begin{itemize}
    \item Our system achieves [X.X] BLEU and [Z]\% glossary adherence on scientific abstracts
    \item Ablation studies confirm the contribution of each component
    \item Comparison with baselines demonstrates improvements in terminology control
\end{itemize}

These results support our thesis contributions:
\begin{enumerate}
    \item Terminology-constrained translation via glossary prompting and enforcement
    \item Document-level coherence through context windows and refinement
    \item Research-grade evaluation framework for systematic analysis
\end{enumerate}


%
% ============================================================================

\chapter{Experimental Evaluation}
\label{ch:experiments}

This chapter presents the experimental evaluation of our document translation system. We describe our experimental setup, present main results, analyze component contributions through ablation studies, and compare with baseline systems.

% ============================================================================
\section{Experimental Setup}
\label{sec:setup}
% ============================================================================

\subsection{Dataset}

We evaluate our system on a corpus of scientific abstracts in the English-French language pair. The corpus consists of:

\begin{itemize}
    \item \textbf{Source}: English scientific abstracts from [describe your sources]
    \item \textbf{Reference}: Professional French translations
    \item \textbf{Size}: [N] documents, [M] segments
    \item \textbf{Domain}: Machine learning, natural language processing
\end{itemize}

% If you have corpus statistics, uncomment and fill in:
% \begin{table}[h]
% \centering
% \begin{tabular}{lc}
% \toprule
% \textbf{Statistic} & \textbf{Value} \\
% \midrule
% Total documents & XXX \\
% Total segments & XXX \\
% Average segments per document & X.X \\
% Average source length (tokens) & XXX \\
% \bottomrule
% \end{tabular}
% \caption{Corpus statistics.}
% \label{tab:corpus-stats}
% \end{table}

\subsection{Evaluation Metrics}

We use the following metrics to evaluate translation quality:

\begin{itemize}
    \item \textbf{BLEU} \cite{papineni2002bleu}: Measures n-gram overlap with reference translations
    \item \textbf{chrF++} \cite{popovic2017chrf}: Character-level F-score, more robust for morphologically rich languages
    \item \textbf{Glossary Adherence}: Percentage of domain terms correctly translated according to our glossary
    \item \textbf{Placeholder Preservation}: Percentage of protected content (formulas, URLs) preserved intact
\end{itemize}

\subsection{System Configuration}

Our system was configured with the following settings:

\begin{itemize}
    \item \textbf{Translation backend}: [GPT-4 / DeepSeek / etc.]
    \item \textbf{Glossary}: [N] domain-specific term pairs
    \item \textbf{Context window}: [N] previous segments
    \item \textbf{Refinement}: Glossary enforcement + coherence pass
\end{itemize}

% ============================================================================
\section{Main Results}
\label{sec:results}
% ============================================================================

Table~\ref{tab:main-results} presents the main translation quality results comparing our full system with baseline configurations.

% Include generated table
% Main results table
\begin{table}[t]
\centering
\small
\begin{tabular}{@{}lcccc@{}}
\toprule
\textbf{System} & \textbf{BLEU}$\uparrow$ & \textbf{chrF++}$\uparrow$ & \textbf{Gloss.}$\uparrow$ & \textbf{Time} \\
\midrule
\textbf{Scitrans Llms} & \textbf{12.0} & \textbf{41.5} & \textbf{100\%} & 1.3s \\
\textbf{Full} & \textbf{12.0} & \textbf{41.5} & \textbf{100\%} & 1.1s \\
No Glossary & 13.2 & 38.4 & 10\% & 0.0s \\
No Context & 12.0 & 41.5 & 100\% & 1.1s \\
No Refinement & 13.2 & 38.4 & 10\% & 0.0s \\
No Masking & 12.0 & 41.5 & 100\% & 1.1s \\
Minimal & 13.2 & 38.4 & 10\% & 0.0s \\
\bottomrule
\end{tabular}
\caption{Translation quality results on the test corpus. Best results in bold.}
\label{tab:main-results}
\end{table}

Our system achieves [X.X] BLEU and [X.X] chrF++ scores, representing an improvement of [Y.Y] BLEU points over the baseline. Notably, glossary adherence reaches [Z]\%, demonstrating effective terminology control.

\subsection{Qualitative Analysis}

Beyond automatic metrics, we observe several qualitative improvements:

\begin{enumerate}
    \item \textbf{Consistent terminology}: Technical terms are translated consistently throughout documents
    \item \textbf{Formula preservation}: Mathematical expressions are preserved without corruption
    \item \textbf{Coherent references}: Pronouns and demonstratives maintain correct referents across sentences
\end{enumerate}

% Example translations
% \begin{table}[h]
% \centering
% \small
% \begin{tabular}{p{0.45\textwidth}p{0.45\textwidth}}
% \toprule
% \textbf{Source} & \textbf{Translation} \\
% \midrule
% Deep learning has revolutionized natural language processing. & L'apprentissage profond a révolutionné le traitement automatique du langage naturel. \\
% \bottomrule
% \end{tabular}
% \caption{Example translation showing terminology consistency.}
% \label{tab:example-translation}
% \end{table}

% ============================================================================
\section{Ablation Study}
\label{sec:ablation}
% ============================================================================

To understand the contribution of each component, we conduct an ablation study by systematically disabling individual features.

% Include generated ablation table
% Ablation study table
\begin{table}[t]
\centering
\small
\begin{tabular}{@{}lccc@{}}
\toprule
\textbf{Configuration} & \textbf{BLEU} & \textbf{$\Delta$BLEU} & \textbf{Gloss.} \\
\midrule
\textbf{Full System} & \textbf{12.0} & --- & \textbf{100\%} \\
-- Glossary & 13.2 & +1.3 & 10\% \\
-- Context & 12.0 & +0.0 & 100\% \\
-- Refinement & 13.2 & +1.3 & 10\% \\
-- Masking & 12.0 & +0.0 & 100\% \\
Minimal & 13.2 & +1.3 & 10\% \\
\bottomrule
\end{tabular}
\caption{Ablation study results. $\Delta$BLEU shows difference from full system.}
\label{tab:ablation}
\end{table}

\subsection{Component Analysis}

Table~\ref{tab:contribution} quantifies each component's contribution to overall quality improvement.

% Include generated contribution table
\input{results/thesis/contribution_table}

Key findings:

\begin{itemize}
    \item \textbf{Glossary enforcement} contributes the largest improvement (+X.X BLEU), confirming the importance of terminology control for scientific translation
    \item \textbf{Document-level refinement} provides +X.X BLEU improvement by improving coherence
    \item \textbf{Placeholder masking} ensures formula preservation with minimal impact on BLEU
    \item \textbf{Context window} shows modest but consistent improvements in pronoun resolution
\end{itemize}

% ============================================================================
\section{Baseline Comparison}
\label{sec:baselines}
% ============================================================================

We compare our system with existing translation approaches:

\begin{itemize}
    \item \textbf{Google Translate}: Commercial neural MT system
    \item \textbf{Naive LLM}: GPT-4 without glossary or context
    \item \textbf{PDFMathTranslate}: Open-source layout-preserving translator
\end{itemize}

% Include baseline comparison if available
% \input{results/thesis/baselines_table}

Our system outperforms all baselines on glossary adherence while maintaining competitive BLEU scores. The improvement is particularly notable for:

\begin{enumerate}
    \item Technical terminology: [X]\% vs [Y]\% for baseline
    \item Formula preservation: [X]\% vs [Y]\% for baseline
    \item Cross-sentence coherence: [qualitative observation]
\end{enumerate}

% ============================================================================
\section{Error Analysis}
\label{sec:errors}
% ============================================================================

We analyze common error types in our system's output:

\subsection{Terminology Errors}

Despite glossary enforcement, some terms are missed when:
\begin{itemize}
    \item Terms appear in unusual grammatical forms
    \item Context suggests alternative translations
    \item Terms are part of larger phrases not in the glossary
\end{itemize}

\subsection{Coherence Issues}

Remaining coherence issues include:
\begin{itemize}
    \item Long-range dependencies beyond context window
    \item Implicit subjects in complex sentences
    \item Domain-specific discourse conventions
\end{itemize}

% ============================================================================
\section{Discussion}
\label{sec:discussion}
% ============================================================================

Our experiments demonstrate that:

\begin{enumerate}
    \item \textbf{Terminology control is crucial}: Glossary enforcement provides the largest quality improvement for scientific translation
    
    \item \textbf{Document context matters}: Even simple context windows improve coherence, with larger windows providing diminishing returns
    
    \item \textbf{LLM refinement is effective}: Post-translation refinement catches errors missed during initial translation
    
    \item \textbf{Layout preservation is achievable}: Placeholder masking successfully protects formulas and special content
\end{enumerate}

\subsection{Limitations}

Our approach has several limitations:

\begin{itemize}
    \item Requires curated glossaries for optimal performance
    \item LLM API costs may limit large-scale deployment
    \item PDF reconstruction quality depends on source document complexity
\end{itemize}

\subsection{Future Work}

Promising directions for future research include:

\begin{itemize}
    \item Automatic glossary extraction from parallel corpora
    \item Fine-tuning smaller models for specific domains
    \item Multi-document translation with shared context
    \item Interactive refinement with human feedback
\end{itemize}

% ============================================================================
\section{Summary}
\label{sec:summary}
% ============================================================================

This chapter presented a comprehensive evaluation of our document translation system. Key findings include:

\begin{itemize}
    \item Our system achieves [X.X] BLEU and [Z]\% glossary adherence on scientific abstracts
    \item Ablation studies confirm the contribution of each component
    \item Comparison with baselines demonstrates improvements in terminology control
\end{itemize}

These results support our thesis contributions:
\begin{enumerate}
    \item Terminology-constrained translation via glossary prompting and enforcement
    \item Document-level coherence through context windows and refinement
    \item Research-grade evaluation framework for systematic analysis
\end{enumerate}

